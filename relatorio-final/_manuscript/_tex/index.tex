% Options for packages loaded elsewhere
% Options for packages loaded elsewhere
\PassOptionsToPackage{unicode}{hyperref}
\PassOptionsToPackage{hyphens}{url}
\PassOptionsToPackage{dvipsnames,svgnames,x11names}{xcolor}
%
\documentclass[
  spanish,
  12pt,
]{article}
\usepackage{xcolor}
\usepackage[top=20mm,left=20mm,bottom=20mm,right=20mm]{geometry}
\usepackage{amsmath,amssymb}
\setcounter{secnumdepth}{5}
\usepackage{iftex}
\ifPDFTeX
  \usepackage[T1]{fontenc}
  \usepackage[utf8]{inputenc}
  \usepackage{textcomp} % provide euro and other symbols
\else % if luatex or xetex
  \usepackage{unicode-math} % this also loads fontspec
  \defaultfontfeatures{Scale=MatchLowercase}
  \defaultfontfeatures[\rmfamily]{Ligatures=TeX,Scale=1}
\fi
\usepackage{lmodern}
\ifPDFTeX\else
  % xetex/luatex font selection
  \setmonofont[]{Consolas}
\fi
% Use upquote if available, for straight quotes in verbatim environments
\IfFileExists{upquote.sty}{\usepackage{upquote}}{}
\IfFileExists{microtype.sty}{% use microtype if available
  \usepackage[]{microtype}
  \UseMicrotypeSet[protrusion]{basicmath} % disable protrusion for tt fonts
}{}
\makeatletter
\@ifundefined{KOMAClassName}{% if non-KOMA class
  \IfFileExists{parskip.sty}{%
    \usepackage{parskip}
  }{% else
    \setlength{\parindent}{0pt}
    \setlength{\parskip}{6pt plus 2pt minus 1pt}}
}{% if KOMA class
  \KOMAoptions{parskip=half}}
\makeatother
% Make \paragraph and \subparagraph free-standing
\makeatletter
\ifx\paragraph\undefined\else
  \let\oldparagraph\paragraph
  \renewcommand{\paragraph}{
    \@ifstar
      \xxxParagraphStar
      \xxxParagraphNoStar
  }
  \newcommand{\xxxParagraphStar}[1]{\oldparagraph*{#1}\mbox{}}
  \newcommand{\xxxParagraphNoStar}[1]{\oldparagraph{#1}\mbox{}}
\fi
\ifx\subparagraph\undefined\else
  \let\oldsubparagraph\subparagraph
  \renewcommand{\subparagraph}{
    \@ifstar
      \xxxSubParagraphStar
      \xxxSubParagraphNoStar
  }
  \newcommand{\xxxSubParagraphStar}[1]{\oldsubparagraph*{#1}\mbox{}}
  \newcommand{\xxxSubParagraphNoStar}[1]{\oldsubparagraph{#1}\mbox{}}
\fi
\makeatother


\usepackage{longtable,booktabs,array}
\usepackage{calc} % for calculating minipage widths
% Correct order of tables after \paragraph or \subparagraph
\usepackage{etoolbox}
\makeatletter
\patchcmd\longtable{\par}{\if@noskipsec\mbox{}\fi\par}{}{}
\makeatother
% Allow footnotes in longtable head/foot
\IfFileExists{footnotehyper.sty}{\usepackage{footnotehyper}}{\usepackage{footnote}}
\makesavenoteenv{longtable}
\usepackage{graphicx}
\makeatletter
\newsavebox\pandoc@box
\newcommand*\pandocbounded[1]{% scales image to fit in text height/width
  \sbox\pandoc@box{#1}%
  \Gscale@div\@tempa{\textheight}{\dimexpr\ht\pandoc@box+\dp\pandoc@box\relax}%
  \Gscale@div\@tempb{\linewidth}{\wd\pandoc@box}%
  \ifdim\@tempb\p@<\@tempa\p@\let\@tempa\@tempb\fi% select the smaller of both
  \ifdim\@tempa\p@<\p@\scalebox{\@tempa}{\usebox\pandoc@box}%
  \else\usebox{\pandoc@box}%
  \fi%
}
% Set default figure placement to htbp
\def\fps@figure{htbp}
\makeatother


% definitions for citeproc citations
\NewDocumentCommand\citeproctext{}{}
\NewDocumentCommand\citeproc{mm}{%
  \begingroup\def\citeproctext{#2}\cite{#1}\endgroup}
\makeatletter
 % allow citations to break across lines
 \let\@cite@ofmt\@firstofone
 % avoid brackets around text for \cite:
 \def\@biblabel#1{}
 \def\@cite#1#2{{#1\if@tempswa , #2\fi}}
\makeatother
\newlength{\cslhangindent}
\setlength{\cslhangindent}{1.5em}
\newlength{\csllabelwidth}
\setlength{\csllabelwidth}{3em}
\newenvironment{CSLReferences}[2] % #1 hanging-indent, #2 entry-spacing
 {\begin{list}{}{%
  \setlength{\itemindent}{0pt}
  \setlength{\leftmargin}{0pt}
  \setlength{\parsep}{0pt}
  % turn on hanging indent if param 1 is 1
  \ifodd #1
   \setlength{\leftmargin}{\cslhangindent}
   \setlength{\itemindent}{-1\cslhangindent}
  \fi
  % set entry spacing
  \setlength{\itemsep}{#2\baselineskip}}}
 {\end{list}}
\usepackage{calc}
\newcommand{\CSLBlock}[1]{\hfill\break\parbox[t]{\linewidth}{\strut\ignorespaces#1\strut}}
\newcommand{\CSLLeftMargin}[1]{\parbox[t]{\csllabelwidth}{\strut#1\strut}}
\newcommand{\CSLRightInline}[1]{\parbox[t]{\linewidth - \csllabelwidth}{\strut#1\strut}}
\newcommand{\CSLIndent}[1]{\hspace{\cslhangindent}#1}

\ifLuaTeX
\usepackage[bidi=basic]{babel}
\else
\usepackage[bidi=default]{babel}
\fi
% get rid of language-specific shorthands (see #6817):
\let\LanguageShortHands\languageshorthands
\def\languageshorthands#1{}


\setlength{\emergencystretch}{3em} % prevent overfull lines

\providecommand{\tightlist}{%
  \setlength{\itemsep}{0pt}\setlength{\parskip}{0pt}}



 


\usepackage{authblk}
\makeatletter
\@ifpackageloaded{caption}{}{\usepackage{caption}}
\AtBeginDocument{%
\ifdefined\contentsname
  \renewcommand*\contentsname{Tabla de contenidos}
\else
  \newcommand\contentsname{Tabla de contenidos}
\fi
\ifdefined\listfigurename
  \renewcommand*\listfigurename{Listado de Figuras}
\else
  \newcommand\listfigurename{Listado de Figuras}
\fi
\ifdefined\listtablename
  \renewcommand*\listtablename{Listado de Tablas}
\else
  \newcommand\listtablename{Listado de Tablas}
\fi
\ifdefined\figurename
  \renewcommand*\figurename{Figura}
\else
  \newcommand\figurename{Figura}
\fi
\ifdefined\tablename
  \renewcommand*\tablename{Tabla}
\else
  \newcommand\tablename{Tabla}
\fi
}
\@ifpackageloaded{float}{}{\usepackage{float}}
\floatstyle{ruled}
\@ifundefined{c@chapter}{\newfloat{codelisting}{h}{lop}}{\newfloat{codelisting}{h}{lop}[chapter]}
\floatname{codelisting}{Listado}
\newcommand*\listoflistings{\listof{codelisting}{Listado de Listados}}
\usepackage{amsthm}
\theoremstyle{definition}
\newtheorem{example}{Ejemplo}[section]
\theoremstyle{definition}
\newtheorem{definition}{Definición}[section]
\theoremstyle{remark}
\AtBeginDocument{\renewcommand*{\proofname}{Prueba}}
\newtheorem*{remark}{Observación}
\newtheorem*{solution}{Solución}
\newtheorem{refremark}{Observación}[section]
\newtheorem{refsolution}{Solución}[section]
\makeatother
\makeatletter
\makeatother
\makeatletter
\@ifpackageloaded{caption}{}{\usepackage{caption}}
\@ifpackageloaded{subcaption}{}{\usepackage{subcaption}}
\makeatother
\usepackage{bookmark}
\IfFileExists{xurl.sty}{\usepackage{xurl}}{} % add URL line breaks if available
\urlstyle{same}
\hypersetup{
  pdftitle={Aproximações para a duração de passeios aleatórios},
  pdflang={es},
  pdfkeywords={Monte Carlo, Julia, Estimação},
  colorlinks=true,
  linkcolor={blue},
  filecolor={Maroon},
  citecolor={Blue},
  urlcolor={Blue},
  pdfcreator={LaTeX via pandoc}}


\title{Aproximações para a duração de passeios aleatórios}
\author{Orientando: Gustavo Silva Garone \and Orientadora: Elisabeti
Kira}
\date{17 de agosto de 2025}
\begin{document}
\maketitle
\begin{abstract}
Inspirados na física, propusemos aproximações para a duração de passeios
aleatórios diversos, em especial, os de difícil derivação analítica.
Para testar as aproximações, utilizamos simulações de Monte Carlo na
linguagem Julia. Analisamos o desempenho desta linguagem comparando-a
com a linguagem Python. Com modelos (passeios) mais sofisticados,
utilizamos recursos computacionais para solução de sistemas lineares na
construção de novas aproximações. Descrevemos o uso de paralelismo e de
outras ferramentas. Exibimos e analisamos os resultados obtidos.
\end{abstract}


\section{Introdução}\label{introduuxe7uxe3o}

Neste artigo, estudaremos a duração de passeios aleatórios e suas
aproximações. Temos como objetivo descrever modeleos de passeios como a
``Ruína do Jogador'' de Pascal, discutido em EDWARDS (1983),
genericamente.

\section{Modelando paseios
aleatórios}\label{modelando-paseios-aleatuxf3rios}

\begin{definition}[Modelo
discreto]\protect\hypertarget{def-modelodiscreto}{}\label{def-modelodiscreto}

Sejam \(\pmb{V}\) um produto cartesiano em \(\mathbb{Z}^d\) de \(d\)
vetores de dimensões \(d_1, d_2, \dots d_d\) da forma
\(v_j =(0, 1, 2, \dots, d_j)\), \(j =
1,2,\ldots d\), \(a\) em \(\mathbb{Z}^d\) um vetor \(d\)-dimensional e
\(\mathbb{X} =
\{X_{v}\}_{\forall v \in \pmb{V}}\) um produto cartesiano de dimensão
igual a \(\pmb{V}\) composto por vetores aleatórios de dimensão \(d\) em
\((\Omega,
\mathscr{F}, P)\). A ênupla \((\pmb{V}, a, \mathbb{X})\) define um
passeio aleatório \(J(\pmb{V},a, \mathbb{X})\) descrito pela sequência
\(\{X_{S_0, 1},
X_{S_1, 2}, \dots X_{S_{t-1}, t}\} = \{X_{v, i}\}{i\geq1}\). Definem-se
\(S_n = a
+\sum^n_{i=1} X_{\cdot, i}\) as posições do passeio no instante \(i\)
baseado no início \(a\) e \(X_{v, i}\) as ``regras do jogo'' na posição
\(v = (v_1, v_2, \dots,
v_d)\) e instante \(i\). Seja
\(\tau = \min\{i : \exists j \in S_i \leq 0 \lor
\exists k \in S_i \geq d_j,\ j = 1, 2, \dots, d\}\) a regra de parada do
passeio.

\end{definition}

\begin{refremark}
Em especial, se \(X_{v, i} = X_{v', i}\) para todo \(v \neq v'\),
dizemos que o passeio está em meio uniforme (uniforme no espaço).
Analogamente, se \(X_{v, i} = X_{v, i'}\) para todo \(i \neq i'\),
dizemos que o passeio é uniforme no tempo.

Simplificaremos casos em meio uniforme escrevendo simplesmente
\(X_{i}\). Não trateremos de casos não uniformes no tempo neste artigo.

\label{rem-uniforme}

\end{refremark}

Para ajudar no entendimento deste modelo admitivelmente complexo,
oferecemos um exemplo:

\begin{example}[Ruína do Jogador
bidimensional]\protect\hypertarget{exm-modelodiscreto}{}\label{exm-modelodiscreto}

Considere um jogador apostando em um casino no lançamento de duas moedas
honestas. A primeira moeda define se apostará seus dólares ou seus
euros. A segunda moeda decide se ganhará ou perderá a jogada. No caso de
derrota, deverá deixar um dólar ou um euro com o casino, enquanto ganha
essa quantidade no caso de sucesso em sua aposta. Terminará o jogo
quando atingir cem dólares ou cem euros, ou caso vá à ruína (seu saldo
chegue em zero em alguma das moedas). O jogador começa com 50 dólares e
25 euros.

Podemos modelar esse jogo no modelo da
Definición~\ref{def-modelodiscreto}, utilizando da simplificação do
Remark~\ref{rem-uniforme}. Tome \(\pmb{V} = (v_1, v_2)\) com
\(v_1 = v_2 = (0, 1, \dots, 100)\), \(a = (50, 25)\) e

\[
R = \begin{cases}
(-1, 0), & 0.25 \\
(0, -1), & 0.25 \\
(1, 0), & 0.25 \\
(0, 1), & 0.25.
\end{cases}
\]

Após \(5\) jogadas, o jogo tomou o seguinte percurso: \[
\begin{aligned}
J(\pmb{V}, a, \pmb{x}) &= \{x_1, x_2, x_3, x_4, x_5\} \\
&= \{(1,0), (-1, 0), (0, 1), (0, 1), (-1, 0)\}.
\end{aligned}
\]

\end{example}

Conforme a regra de parada \(\tau\), o jogo encerrar-se-á assim que um
elemento de \(S_n\), isto é, uma moeda (dólar ou euro) chegar em zero ou
cem. No Ejemplo~\ref{exm-modelodiscreto}, é fácil ver que o jogo ainda
não acabou, pois \(S_5 = (24,
52)\).

Estamos interessados na duração esperada do jogo, ou seja, queremos
encontrar \(E(\tau)\). Abordaremos técnicas analíticas, computacionais e
estimativas para este valor no restante deste artigo.

\newpage{}

\section{Referências}\label{referuxeancias}

\phantomsection\label{refs}
\begin{CSLReferences}{0}{1}
\bibitem[\citeproctext]{ref-edwards_pascals_1983}
EDWARDS, A. W. F. \href{https://doi.org/10.2307/1402732}{Pascal's
Problem: The 'Gambler's Ruin'}. \textbf{International Statistical Review
/ Revue Internationale de Statistique}, v. 51, n. 1, p. 73-79, 1983.

\end{CSLReferences}




\end{document}
