\documentclass[a4paper,10pt,twocolumn]{article}

%Layout
% \setlength\columnsep{.8cm}
% \setlength\topmargin{-1.71cm}
% \setlength\textheight{23.4cm}
% \setlength\textwidth{15.8cm}

%Pacotes
\usepackage{amsthm,amssymb,amsmath}
\usepackage[english]{babel}
\usepackage[utf8]{inputenc}
\usepackage{amsfonts}
\usepackage{graphicx}
\usepackage{float}
\usepackage{ragged2e}
\usepackage{fancyhdr}
\usepackage{titlesec}
\usepackage[font=small]{caption}
\usepackage[
layout = a4paper,
includehead,
paperheight = 29.7cm,
headheight = 3.41cm,
paperwidth = 21cm,
textwidth = 15.8cm,
textheight = 23.4cm,
columnsep = 0.8cm,
top = 1.25cm,
bottom = 4.2cm,
left = 2.6cm,
right = 2.6cm,
headsep = 1.06cm,
]{geometry}

%Fonte Arial
\renewcommand{\rmdefault}{phv}
\renewcommand{\sfdefault}{phv}

\titleformat
{\section} % command
[display] % shape
{\normalfont\fontsize{13}{13}\bfseries} % format
{} % label
{0cm} % sep
{
    \centering
} % before-code
[
] % after-code
\titlespacing*{\section}
{0pt}
{2pt}
{2pt}
[0pt]

\begin{document}
\pagestyle{fancy}
\fancyhead{}
\fancyhead[L]{\includegraphics[width=3.41cm]{logo_siicusp.jpg}}
\renewcommand{\headrulewidth}{0pt}
\pagenumbering{gobble}

\twocolumn[{
    % Bloco centralizado (ocupa as duas colunas)
    \begin{center}
        {\fontsize{13}{13}\selectfont \textbf{Estimating the duration of random
        walks}}\\[1em]
        {\fontsize{13}{13}\selectfont \textbf{Eduardo Yukio Garrafa Ishihara}}\\[0.3em]
        {\fontsize{13}{13}\selectfont \textbf{Gustavo Silva Garone}}\\[0.9em]
        {\fontsize{13}{13}\selectfont \textbf{Prof. Elisabeti Kira, Ph.D}}\\[0.8em]
        {\fontsize{13}{13}\selectfont Institute of Mathematics and Statistics,
        University of São Paulo}\\[0.3em]
        {\fontsize{10}{10}\selectfont eduardoyukio.ishihara@usp.br}
    \end{center}
    \vspace{1.5em} % espaço antes de começar o corpo do texto
}]

\section{Objectives}

The Gambler’s Ruin is a random walk, one of the most classical models of a
Markovian stochastic process. In this model, a gambler wagers money in each
round and, according to a probabilistic rule, either gains or loses varying
amounts. Such models are widely applicable in statistical modeling and
computational algorithms \cite{ross_markov_2019}. Their behavior has been
thoroughly studied in the simplest form — the gambler wins
$\mathrm{R}\$1.00$ with probability $p$ and loses $\mathrm{R}\$1.00$ with
probability $1-p$ — particularly concerning the probability of ruin or
absorption \cite{ross_introduction_2019}. The aim of this work is to investigate
the expected time to ruin in more complex models, by proposing methods to
estimate this duration and evaluating their performance through computational
approaches, thus expanding the understanding of these processes and some of
their generalizations.

\section{Materials and methods}

We analyzed the calculation of the expectation and variance of the duration
of the classical Gambler’s Ruin problem \cite{andel_variance_2012}, highlighting
the use of systems of finite difference equations. We developed an initial heuristic
estimator based on an analogy with average velocity in Physics, that yielded
accurate approximations for most random walks, both in the simple case and in
models where the gambler may win or lose amounts different from
$\mathrm{R}\$1.00$ per round. In the fair game case, i.e., when the expected gain
per round is zero, the estimator deviated significantly from the theoretical and
simulated values.

For random walks with more general rules, where analytical solutions are not
available, we evaluated this estimator using Monte Carlo simulations, based on
methods proposed by Ritter \cite{ritter_determining_2011}, using the Python
programming language. As we considered broader rules and rules that greatly
extended the simulation times and required more computational power, we
migrated to the Julia language, as recommended by Godoy
\cite{godoy_evaluating_2023}. We further refined the estimator to account for
ties within a round.

Having identified the limitations of this estimator in one dimension, we extended
our study to random walks in $\mathbb{Z}^d$. By solving systems of equations
computationally with symbolic libraries, we obtained analytical results for unit-step
walks with different probabilistic rules. New estimators were proposed and analyzed
for both unit and non-unit step walks, which provided good estimates when the
number of dimensions was sufficiently large or when the probabilistic rule of the
model was non-pathological.

\section{Results}

The proposed estimator showed efficacy in predicting the duration of
one-dimensional uniform walks, as depicted by Figure 1, and adapts successfully
when the transition probability depends on the gambler's current fortune.

\begin{figure}[H]
    \centering
    \includegraphics[width=7.5cm]{"graficoest-en.pdf"}
    \centering
    \caption{Comparing the estimator with 100.000 simulations per starting
    position}
\end{figure}

In higher dimensions the expected duration of the random walk must be obtained
numerically due to the system of balance equations growing exponentially with
the number of dimensions, rendering a closed form solution intractable. Still, a
surprising result was revealed: once the dimensionality exceeds roughly the
$100$th dimension, the average time to game completion increases almost linearly
with each additional dimension as shown in Figure 2.

\begin{figure}[h]
    \centering
    \includegraphics[width=7.5cm]{"boxplots-en.pdf"}
    \centering
    \caption{Simulated steps in 1.000 simulations em different dimensions.}
\end{figure}

In the particular model where the gambler cannot lose his money, we've obtained
analytic expressions for the uniform, non-uniform, random and non-random
variants. The complexity of the equations that describe the game's duration
grows proportionally to the state space's size. In all cases, the estimated
values show great proximity to simulated ones, although greater discrepancies
show when the game is close to fair - that is, when the gambler's expected
earnings per round are close to $0$ - or when the starting position is close to
a boundary value.

\section{Conclusions}

The Gambler's Ruin is a classical math problem proposed almost four centuries
ago. Through intuition, undergraduates can formulate fresh questions and
hypothesis, explore them computationally, and finally developing rigorous
mathematics. It is our hope that this work encourages other undergraduates to
explore and reexplore classical math problems in search of solutions to modern
day challenges.

The authors declare no conflict of interests. Eduardo Yukio conceived the
theoretical framework behind the estimators and derived the paper's analytical
results. Gustavo Garone provided computational support and implemented the
simulations used for improving the estimators.

\section{Acknowledgements}

We would like to thank our advisor, Elisabeti Kira, for her patience, trust, and
mentorship that enabled this project to be completed. We would also like to thank
the University of São Paulo for financing our research through the Programa
Unificado de Bolsas (PUB).

\vspace{0.5cm}
\bibliographystyle{plain}
\bibliography{references}

\end{document}
