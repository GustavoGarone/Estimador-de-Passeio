\documentclass[a4paper,10pt,twocolumn]{article}

%Layout
% \setlength\columnsep{.8cm}
% \setlength\topmargin{-1.71cm}
% \setlength\textheight{23.4cm}
% \setlength\textwidth{15.8cm}

%Pacotes
\usepackage{amsthm,amssymb,amsmath}
\usepackage[english]{babel}
\usepackage[utf8]{inputenc}
\usepackage{amsfonts}
\usepackage{graphicx}
\usepackage{float}
\usepackage{ragged2e}
\usepackage{fancyhdr}
\usepackage{titlesec}
\usepackage[font=small]{caption}
\usepackage[
layout = a4paper,
includehead,
paperheight = 29.7cm,
headheight = 3.41cm,
paperwidth = 21cm,
textwidth = 15.8cm,
textheight = 23.4cm,
columnsep = 0.8cm,
top = 1.25cm,
bottom = 4.2cm,
left = 2.6cm,
right = 2.6cm,
headsep = 1.06cm,
]{geometry}

%Fonte Arial
\renewcommand{\rmdefault}{phv}
\renewcommand{\sfdefault}{phv}

\titleformat
{\section} % command
[display] % shape
{\normalfont\fontsize{13}{13}\bfseries} % format
{} % label
{0cm} % sep
{
    \centering
} % before-code
[
] % after-code
\titlespacing*{\section}
{0pt}
{2pt}
{2pt}
[0pt]

\begin{document}
\pagestyle{fancy}
\fancyhead{}
\fancyhead[L]{\includegraphics[width=3.41cm]{logo_siicusp.jpg}}
\renewcommand{\headrulewidth}{0pt}
\pagenumbering{gobble}

\twocolumn[{
    % Bloco centralizado (ocupa as duas colunas)
    \begin{center}
        {\fontsize{13}{13}\selectfont \textbf{Estimating the duration of random
        walks}}\\[1em]
        {\fontsize{13}{13}\selectfont \textbf{Eduardo Yukio Garrafa Ishihara}}\\[0.3em]
        {\fontsize{13}{13}\selectfont \textbf{Gustavo Silva Garone}}\\[0.9em]
        {\fontsize{13}{13}\selectfont \textbf{Prof. Elisabeti Kira, Ph.D}}\\[0.8em]
        {\fontsize{13}{13}\selectfont Institute of Mathematics and Statistics,
        University of São Paulo}\\[0.3em]
        {\fontsize{10}{10}\selectfont eduardoyukio.ishihara@usp.br}
    \end{center}
    \vspace{1.5em} % espaço antes de começar o corpo do texto
}]

\section{Objectives}

A Ruína do Jogador é um passeio aleatório, um dos mais conhecidos modelos de
processo estocástico Markoviano. Nele, um jogador aposta seu dinheiro a cada
rodada e, com base em uma regra probabilística, ganha ou perde valores diversos.
Modelos desse tipo possuem grande utilidade na modelagem estatística e
aplicações em algoritmos computacionais \cite{ross_markov_2019}. Seu
comportamento já foi amplamente estudado em sua forma mais simples - o jogador
ganha $\mathrm{R}\$1,00$ com probabilidade $p$ e perde $\mathrm{R}\$1,00$ com
probabilidade $1-p$ - em especial no que diz respeito à probabilidade de ruína
ou vitória \cite{ross_introduction_2019}. Esse trabalho tem como objetivo
estudar o tempo esperado até a ruína para modelos mais complexos, propondo
métodos de estimar essa duração e avaliando-os por vias computacionais, assim
expandindo o conhecimento acerca desses processos e algumas de suas variações.

\section{Materials and methods}

Analisamos o cálculo da esperança e da variância da duração da Ruína do Jogador
simples \cite{andel_variance_2012}, na qual destacou-se o uso de sistemas de
equações de diferenças finitas. Desenvolvemos um primeiro estimador heurístico
baseado numa analogia à velocidade média em Física, que forneceu boas
estimativas  para a maioria dos passeios aleatórios, tanto os simples como
aqueles em que os jogadores podem ganhar ou perder valores diferentes de
$\mathrm{R}\$1,00$ por rodada. No caso do jogo honesto, isso é, quando o lucro
esperado em cada rodada é nulo, o estimador forneceu estimativas que desviaram
significativamente dos valores calculados e simulados.

Para passeios com regras mais amplas que não possuem resultado analítico,
avaliamos este estimador por simulações de Monte Carlo, com base em métodos
propostos por Ritter \cite{ritter_determining_2011}, na linguagem de programação
Python. Conforme considerávamos regras de jogo mais amplas e que prolongavam
muito o tempo das simulações e demandavam maior poder computacional, passamos a
empregar a linguagem Julia, como recomendado por Godoy
\cite{godoy_evaluating_2023}. Refinamos o estimador para lidar com situações de
empate dentro da rodada.

Compreendidos os limites deste estimador para uma dimensão, estendemos nossos
estudos para passeios em $\mathbb{Z}^d$. Resolvendo sistemas de equações
computacionalmente por bibliotecas simbólicas, conseguimos resultados analíticos
para passeios unitários e com diferentes regras probabilísticas. Novos
estimadores foram propostos e analisados para passeios unitários ou não, os
quais apresentaram boas estimativas quando o número de dimensões é
suficientemente grande ou a regra probabilística do modelo não é patológica.

\section{Results}

The proposed estimator showed efficacy in predicting the duration of
one-dimensional uniform walks, as depicted by Figure 1, and adapts successfully
when the transition probability depends on the gambler's current fortune.

\begin{figure}[H]
    \centering
    \includegraphics[width=7.5cm]{"graficoest-en.pdf"}
    \centering
    \caption{Comparing the estimator with 100.000 simulations per starting
    position}
\end{figure}

In higher dimensions the expected duration of the random walk must be obtained
numerically due to the system of balance equations growing exponentially with
the number of dimensions, rendering a closed form solution intractable. Still, a
surprising result was revealed: once the dimensionality exceeds roughly the
$100$th dimension, the average time to game completion increases almost linearly
with each additional dimension as shown in Figure 2.

\begin{figure}[h]
    \centering
    \includegraphics[width=7.5cm]{"boxplots-en.pdf"}
    \centering
    \caption{Simulated steps in 1.000 simulations em different dimensions.}
\end{figure}

In the particular model where the gambler cannot lose his money, we've obtained
analytic expressions for the uniform, non-uniform, random and non-random
variants. The complexity of the equations that describe the game's duration
grows proportionally to the state space's size. In all cases, the estimated
values show great proximity to simulated ones, although greater discrepancies
show when the game is close to fair - that is, when the gambler's expected
earnings per round are close to $0$ - or when the starting position is close to
a boundary value.

\section{Conclusions}

The Gambler's Ruin is a classical math problem proposed almost four centuries
ago. Through intuition, undergraduates can formulate fresh questions and
hypothesis, explore them computationally, and finally developing rigorous
mathematics. It is our hope that this work encourages other undergraduates to
explore and reexplore classical math problems in search of solutions to modern
day challenges.

The authors declare no conflict of interests. Eduardo Yukio conceived the
theoretical framework behind the estimators and derived the paper's analytical
results. Gustavo Garone provided computational support and implemented the
simulations used for improving the estimators.

\section{Acknowledgements}

We would like to thank our advisor, Elisabeti Kira, for her patience, trust, and
mentorship that enabled this project to be completed. We would also like to thank
the University of São Paulo for financing our research through the Programa
Unificado de Bolsas (PUB).

\vspace{0.5cm}
\bibliographystyle{plain}
\bibliography{references}

\end{document}
