\documentclass[a4paper,10pt,twocolumn]{article}

%Layout
% \setlength\columnsep{.8cm}
% \setlength\topmargin{-1.71cm}
% \setlength\textheight{23.4cm}
% \setlength\textwidth{15.8cm}

%Pacotes
\usepackage{amsthm,amssymb,amsmath}
\usepackage[brazil]{babel}
\usepackage[utf8]{inputenc}
\usepackage{amsfonts}
\usepackage{graphicx}
\usepackage{float}
\usepackage{ragged2e}
\usepackage{fancyhdr}
\usepackage{titlesec}
\usepackage[font=small]{caption}
\usepackage[
layout = a4paper,
includehead,
paperheight = 29.7cm,
headheight = 3.41cm,
paperwidth = 21cm,
textwidth = 15.8cm,
textheight = 23.4cm,
columnsep = 0.8cm,
top = 1.25cm,
bottom = 4.2cm,
left = 2.6cm,
right = 2.6cm,
headsep = 1.06cm,
]{geometry}

%Fonte Arial
\renewcommand{\rmdefault}{phv}
\renewcommand{\sfdefault}{phv}

\titleformat
{\section} % command
[display] % shape
{\normalfont\fontsize{13}{13}\bfseries} % format
{} % label
{0cm} % sep
{
    \centering
} % before-code
[
] % after-code
\titlespacing*{\section}
{0pt}
{2pt}
{2pt}
[0pt]

\begin{document}
\pagestyle{fancy}
\fancyhead{}
\fancyhead[L]{\includegraphics[width=3.41cm]{logo_siicusp.jpg}}
\renewcommand{\headrulewidth}{0pt}
\pagenumbering{gobble}

\twocolumn[{
    % Bloco centralizado (ocupa as duas colunas)
    \begin{center}
        {\fontsize{13}{13}\selectfont \textbf{Estimando a duração de passeios
        aleatórios}}\\[1em]
        {\fontsize{13}{13}\selectfont \textbf{Eduardo Yukio Garrafa Ishihara}}\\[0.3em]
        {\fontsize{13}{13}\selectfont \textbf{Gustavo Silva Garone}}\\[0.9em]
        {\fontsize{13}{13}\selectfont \textbf{Profª. Dra. Elisabeti Kira}}\\[0.8em]
        {\fontsize{13}{13}\selectfont Instituto de Matemática e Estatística, Universidade de São Paulo}\\[0.3em]
        {\fontsize{10}{10}\selectfont eduardoyukio.ishihara@usp.br}
    \end{center}
    \vspace{1.5em} % espaço antes de começar o corpo do texto
}]

\section{Objetivos}

A Ruína do Jogador é um passeio aleatório, um dos mais conhecidos modelos de
processo estocástico Markoviano. Nele, um jogador aposta seu dinheiro a cada
rodada e, com base em uma regra probabilística, ganha ou perde valores diversos.
Modelos desse tipo possuem grande utilidade na modelagem estatística e
aplicações em algoritmos computacionais \cite{ross_markov_2019}. Seu
comportamento já foi amplamente estudado em sua forma mais simples - o jogador
ganha $\mathrm{R}\$1,00$ com probabilidade $p$ e perde $\mathrm{R}\$1,00$ com
probabilidade $1-p$ - em especial no que diz respeito à probabilidade de ruína
ou vitória \cite{ross_introduction_2019}. Esse trabalho tem como objetivo
estudar o tempo esperado até a ruína para modelos mais complexos, propondo
métodos de estimar essa duração e avaliando-os por vias computacionais, assim
expandindo o conhecimento acerca desses processos e algumas de suas variações.

\section{Métodos}

Analisamos o cálculo da esperança e da variância da duração da Ruína do Jogador
simples \cite{andel_variance_2012}, na qual destacou-se o uso de sistemas de
equações de diferenças finitas. Desenvolvemos um primeiro estimador heurístico
baseado numa analogia à velocidade média em Física, que forneceu boas
estimativas  para a maioria dos passeios aleatórios, tanto os simples como
aqueles em que os jogadores podem ganhar ou perder valores diferentes de
$\mathrm{R}\$1,00$ por rodada. No caso do jogo honesto, isso é, quando o lucro
esperado em cada rodada é nulo, o estimador forneceu estimativas que desviaram
significativamente dos valores calculados e simulados.

Para passeios com regras mais amplas que não possuem resultado analítico,
avaliamos este estimador por simulações de Monte Carlo, com base em métodos
propostos por Ritter \cite{ritter_determining_2011}, na linguagem de programação
Python. Conforme considerávamos regras de jogo mais amplas e que prolongavam
muito o tempo das simulações e demandavam maior poder computacional, passamos a
empregar a linguagem Julia, como recomendado por Godoy
\cite{godoy_evaluating_2023}. Refinamos o estimador para lidar com situações de
empate dentro da rodada.

Compreendidos os limites deste estimador para uma dimensão, estendemos nossos
estudos para passeios em $\mathbb{Z}^d$. Resolvendo sistemas de equações
computacionalmente por bibliotecas simbólicas, conseguimos resultados analíticos
para passeios unitários e com diferentes regras probabilísticas. Novos
estimadores foram propostos e analisados para passeios unitários ou não, os
quais apresentaram boas estimativas quando o número de dimensões é
suficientemente grande ou a regra probabilística do modelo não é patológica.

\section{Resultados}

O estimador proposto mostrou-se eficaz para o caso unidimensional em meio
uniforme, como mostra a Figura 1, e foi adaptado com sucesso ao caso em que a
regra probabilística depende da fortuna atual do jogador.

\begin{figure}[H]
    \centering
    \includegraphics[width=7.5cm]{"graficoest.pdf"}
    \centering
    \caption{Comparação do estimador com 100.000 simulações por posição inicial}
\end{figure}

Em dimensões mais altas, constatou-se que a duração esperada do passeio só pode
ser determinada por métodos numéricos, dado o crescimento exponencial das
equações envolvidas com a dimensão. Surpreendentemente, observou-se crescimento
linear do tempo médio de duração do jogo conforme o número de dimensões aumenta
a partir da $100$ª dimensão, como ilustra a Figura 2.

\begin{figure}[h]
    \centering
    \includegraphics[width=7.5cm]{"boxplots.pdf"}
    \centering
    \caption{Passos simulados até fim do jogo com $1000$ simulações em diversas
    dimensões.}
\end{figure}

Para o caso particular do modelo em que o jogador não pode perder dinheiro,
foram obtidas expressões teóricas em meio uniforme, não uniforme, em meio
aleatório e não aleatório. Foi revelado que a complexidade das equações que
descrevem o tempo médio de duração do jogo cresce proporcionalmente ao tamanho
do espaço de estados. Em todos os casos, os estimadores se aproximaram dos
valores simulados, embora apresentem maiores desvios quando o jogo é
aproximadamente honesto, ou quando o estado inicial está próximo das fronteiras.

\section{Conclusões}

A Ruína do Jogador é um problema clássico, proposto há quase quatro séculos.
Ainda assim, munidos apenas da intuição, é possível para alunos da graduação
explorarem novas perguntas e hipóteses, testá-las com ajuda dos recursos
computacionais, hoje abundantes, para finalmente formalizá-las com rigor
matemático. Esperamos que esse trabalho incentive outros alunos da graduação a
explorarem e reexplorarem por si problemas matemáticos clássicos em busca de
soluções para problemas modernos.

\section{Agradecimentos}

Agradecemos, primeiramente, à Professora Elisabeti Kira pela confiança, pela
compreensão e pelos ensinamentos ao orientar-nos durante a produção desse artigo
e à Universidade de São Paulo (USP) pelo apoio financeiro concedido por meio do
Programa Unificado de Bolsas.

\vspace{0.5cm}
\bibliographystyle{plain}
\bibliography{references}

\end{document}
