\documentclass[a4paper,10pt,twocolumn]{article}

%Layout
% \setlength\columnsep{.8cm}
% \setlength\topmargin{-1.71cm}
% \setlength\textheight{23.4cm}
% \setlength\textwidth{15.8cm}

%Pacotes
\usepackage{amsthm,amssymb,amsmath}
\usepackage[brazil]{babel}
\usepackage[utf8]{inputenc}
\usepackage{amsfonts}
\usepackage{graphicx}
\usepackage{ragged2e}
\usepackage{fancyhdr}
\usepackage{titlesec}
\usepackage[
layout = a4paper,
includehead,
paperheight = 29.7cm,
headheight = 3.41cm,
paperwidth = 21cm,
textwidth = 15.8cm,
textheight = 23.4cm,
columnsep = 0.8cm,
top = 1.25cm,
bottom = 4.2cm,
left = 2.6cm,
right = 2.6cm,
headsep = 1.06cm,
]{geometry}

%Fonte Arial
\renewcommand{\rmdefault}{phv}
\renewcommand{\sfdefault}{phv}

\titleformat
{\section} % command
[display] % shape
{\normalfont\fontsize{13}{13}\bfseries} % format
{} % label
{0cm} % sep
{
    \centering
} % before-code
[
] % after-code
\titlespacing*{\section}
{0pt}
{2pt}
{2pt}
[0pt]

\begin{document}
\pagestyle{fancy}
\fancyhead{}
\fancyhead[L]{\includegraphics[width=3.41cm]{logo_siicusp.jpg}}
\renewcommand{\headrulewidth}{0pt}

\twocolumn[{
    % Bloco centralizado (ocupa as duas colunas)
    \begin{center}
        {\fontsize{13}{13}\selectfont \textbf{ESTIMANDO A DURAÇÃO ESPERADA DE
        PASSEIOS ALEATÓRIOS}}\\[1em]
        {\fontsize{13}{13}\selectfont \textbf{Eduardo Yukio Garrafa Ishihara}}\\[0.3em]
        {\fontsize{13}{13}\selectfont \textbf{Gustavo Silva Garone}}\\[0.9em]
        {\fontsize{13}{13}\selectfont \textbf{Prof.ª Dra. Elisabeti Kira}}\\[0.8em]
        {\fontsize{13}{13}\selectfont Instituto de Matemática e Estatística, Universidade de São Paulo}\\[0.3em]
        {\fontsize{10}{10}\selectfont eduardoyukio.ishihara@usp.br}
    \end{center}
    \vspace{1.5em} % espaço antes de começar o corpo do texto
}]

\section{Objetivos}

A Ruína do Jogador é um dos mais conhecidos modelos de passeio aleatório, um
processo estocástico markoviano com espaço de estados enumerável e barreiras
absorventes. Nele, dois jogadores apostam seu dinheiro a cada rodada e, com base
em uma distribuição de probabilidades, ganham ou perdem valores diversos.
Passeios desse tipo possuem grande utilidade na modelagem estatística e
aplicações em algoritmos computacionais \cite{ross_markov_2019}. Por isso, seu
comportamento já foi amplamente estudado em sua forma mais simples (unitário,
uniforme e em meio não aleatório), em especial quanto a probabilidade de
ruína ou vitória \cite{ross_introduction_2019}. Esse trabalho tem como objetivo
estudar o tempo esperado até a ruína para passeios mais complexos, propondo
métodos de estimar essa duração e avaliando-os por vias computacionais, assim
expandindo o conhecimento acerca de passeios desse tipo e algumas de suas
variações.

\section{Métodos}

Iniciamos por uma estudo da literatura sobre a derivação analítica de passeios
mais simples. Analisamos o cálculo da esperança e da variância da duração da
Ruína do Jogador simples \cite{andel_variance_2012} na qual destacou-se o uso de
sistemas de equações de diferenças finitas. Depois, desenvolvemos um primeiro
estimador heurístico baseado numa analogia à velocidade média em Física, que
funcionou para a maioria dos passeios não unitários. Definimos passeios não
unitários como aqueles em que os jogadores podem ganhar ou perder valores
diferentes de $\mathrm{R}\$1,00$ por rodada.

Uma vez que estes passeios não possuem resultado analítico (ou de difícil
construção), avaliamos este estimador por simulações de Monte Carlo, com base em
métodos propostos por Ritter \cite{ritter_determining_2011}, na linguagem de
programação Python. Posteriormente, conforme utilizávamos regras de jogo que
prolongavam muito as simulações e demandavam maior poder computacional, passamos
a empregar a linguagem Julia, como recomendado por Godoy
\cite{godoy_evaluating_2023}. Refinamos o estimador para lidar com situações em
que o empate entre os jogadores (não há troca de dinheiro na rodada) é
possível.

Compreendidos os limites deste estimador, estendemos nossos estudos para
passeios em $\mathbb{Z}^d$, como jogos com mais tipos de moeda. Resolvendo
sistemas de equações computacionalmente por bibliotecas simbólicas, conseguimos
resultados analíticos para passeios unitários, mas não necessariamente uniformes
ou não aleatórios. Novos estimadores foram pensados e analisados para passeios
não unitários, também em dimensões superiores e em meio não uniforme.

\section{Resultados}

O estimador proposto mostrou-se eficaz para o caso unidimensional em meio
uniforme, como mostra a Figura 1, e foi adaptado com sucesso ao caso não
uniforme. Em dimensões mais altas, constatou-se que a duração esperada do
passeio só pode ser determinada numericamente, dada a explosão combinatória das
equações envolvidas e, surpreendentemente, demonstrou crescimento linear
conforme o número de dimensões aumenta a partir da $50$ª dimensão, como consta a
Figura 2. Para o "passeio do foguete", em que o jogador não pode perder
dinheiro, foram obtidas expressões teóricas em meios uniforme, não uniforme, em
meios aleatórios e não aleatórios. Foi revelando que a complexidade cresce
proporcionalmente ao número de estados possíveis. Em todos os casos, os
estimadores se aproximaram dos valores simulados, embora apresentem maiores
desvios quando a esperança da variável de transição se aproxima de zero, o que
denominamos jogos com pouco movimento assintótico esperado, ou quando o estado
inicial está próximo das fronteiras.

\section{Conclusões}

A Ruína do Jogador é um problema clássico de quase quatro séculos atrás. Ainda
assim, mesmo munidos apenas da intuição, é possível para alunos da graduação
explorarem novas perguntas e hipóteses, testá-las com ajuda dos recursos
computacionais hoje abundantes para finalmente formalizá-las com rigor
matemático. Esperamos que esse trabalho incentive outros alunos da graduação a
explorarem e reexplorarem por si clássicos problemas matemáticos em busca de
soluções para problemas modernos.

\section{Agradecimentos}

Agradecemos, primeiramente, à Professora Elisabeti Kira pela confiança, pela
compreensão e pelos ensinamentos ao orientar-nos durante a produção desse artigo
e à Universidade de São Paulo (USP) pelo apoio financeiro concedido por meio do
Programa Unificado de Bolsas.

\vspace{2.5pt}
\bibliographystyle{plain}
\bibliography{references}

\end{document}
