\documentclass[a4paper,10pt,twocolumn]{article}

%Layout
% \setlength\columnsep{.8cm}
% \setlength\topmargin{-1.71cm}
% \setlength\textheight{23.4cm}
% \setlength\textwidth{15.8cm}

%Pacotes
\usepackage{amsthm,amssymb,amsmath}
\usepackage[brazil]{babel}
\usepackage[utf8]{inputenc}
\usepackage{amsfonts}
\usepackage{graphicx}
\usepackage{ragged2e}
\usepackage{fancyhdr}
\usepackage{titlesec}
\usepackage[
layout = a4paper,
includehead,
paperheight = 29.7cm,
headheight = 3.41cm,
paperwidth = 21cm,
textwidth = 15.8cm,
textheight = 23.4cm,
columnsep = 0.8cm,
top = 1.25cm,
bottom = 4.2cm,
left = 2.6cm,
right = 2.6cm,
headsep = 1.06cm,
]{geometry}

%Fonte Arial
\renewcommand{\rmdefault}{phv}
\renewcommand{\sfdefault}{phv}

\titleformat
{\section} % command
[display] % shape
{\normalfont\fontsize{13}{13}\bfseries} % format
{} % label
{0cm} % sep
{
    \centering
} % before-code
[
] % after-code
\titlespacing*{\section}
{0pt}
{2pt}
{2pt}
[0pt]

\begin{document}
\pagestyle{fancy}
\fancyhead{}
\fancyhead[L]{\includegraphics[width=3.41cm]{logo_siicusp.jpg}}
\renewcommand{\headrulewidth}{0pt}

\twocolumn[{
    % Bloco centralizado (ocupa as duas colunas)
    \begin{center}
        {\fontsize{13}{13}\selectfont \textbf{ESTIMANDO A DURAÇÃO ESPERADA DE
        PASSEIOS ALEATÓRIOS}}\\[1em]
        {\fontsize{13}{13}\selectfont \textbf{Eduardo Yukio Garrafa Ishihara}}\\[0.3em]
        {\fontsize{13}{13}\selectfont \textbf{Gustavo Silva Garone}}\\[0.9em]
        {\fontsize{13}{13}\selectfont \textbf{Prof.ª Dra. Elisabeti Kira}}\\[0.8em]
        {\fontsize{13}{13}\selectfont Instituto de Matemática e Estatística, Universidade de São Paulo}\\[0.3em]
        {\fontsize{10}{10}\selectfont eduardoyukio.ishihara@usp.br}
    \end{center}
    \vspace{1.5em} % espaço antes de começar o corpo do texto
}]

\section{Objetivos}

A Ruína do Jogador é um dos mais conhecidos modelos de passeio aleatório, um
processo estocástico markoviano com espaço de estados enumerável e barreiras
absorventes. Nele, dois jogadores apostam seu dinheiro a cada rodada e, com base
em uma distribuição de probabilidades, ganham ou perdem valores diversos. Por sua
utilidade na modelagem  estatística e aplicação em algoritmos computacionais,
seu comportamento já foi amplamente estudado em sua forma mais simples
(unitário, uniforme e em meio não aleatório), em especial quanto a probabilidade
de ruína \cite{ross_introduction_2019}. Esse trabalho tem como objetivo estudar
o tempo esperado até a ruína para passeios mais complexos, propondo métodos de
estimar essa duração e avaliando-os por vias computacionais.

\section{Métodos}

Iniciamos por uma revisão da literatura na derivação analítica de passeios mais
simples. Analisamos o cálculo da esperança e da variância da duração da Ruína do
Jogador simples \cite{andel_variance_2012} na qual destacou-se o uso de sistemas
de equações de diferenças finitas. Depois, desenvolvemos um primeiro estimador
heurístico baseado em analogia à velocidade média em Física, que funcionou para
a maioria dos passeios não unitários. Isto é, passeios em que os jogadores podem
ganhar ou perder valores diferentes de $\mathrm{R}\$1,00$ por rodada. Uma vez que estes
passeios não possuem resultado analítico (ou de difícil construção), avaliamos este
estimador por simulações de Monte Carlo na linguagem de programação Python.
Posteriormente, conforme utilizávamos regras de jogo que prolongavam muito as
simulações e demandavam maior poder computacional, passamos a utilizar a
linguagem Julia por sua velocidade.

Compreendidos os limites deste estimador,

\section{Resultados}

O estimador apresentou resultados muito
próximos dos simulados, como exibe a Figura 1.

O estimador proposto mostrou-se eficaz para o caso unidimensional em meio
uniforme e foi adaptado com sucesso ao caso não uniforme. Em dimensões mais
altas, constatou-se que a esperança do tempo de ruína só pode ser determinada
numericamente, dada a explosão combinatória das equações envolvidas e,
demonstrou crescimento linear conforme o número de dimensões aumenta a partir da
$50$ª dimensão. Para o passeio do foguete, foram obtidas expressões teóricas em
meios uniforme, não uniforme, em meios aleatórios e não aleatórios. Foi
necessário o uso de computadores para obter valores numéricos dos casos
aleatórios não uniformes, revelando que a complexidade cresce proporcionalmente
ao número de estados possíveis. Em todos os casos, os estimadores se aproximaram
dos valores simulados, embora apresentem maiores desvios quando a esperança da
variável de transição se aproxima de zero ou quando o estado inicial está
próximo das fronteiras.

\section{Conclusões}

O estudo demonstrou que o estimador proposto é uma ferramenta versátil para
analisar a duração média da ruína do jogador em diferentes configurações,
especialmente quando resultados teóricos são inviáveis. A extensão para o
passeio do foguete mostrou que modelos alternativos podem ser tratados dentro do
mesmo arcabouço metodológico. No entanto, a análise em altas dimensões e em
meios aleatórios não uniformes permanece um desafio, reforçando a necessidade de
abordagens computacionais e de futuros refinamentos do estimador.

\vspace{2.5pt}
\bibliographystyle{plain}
\bibliography{references}

\end{document}
